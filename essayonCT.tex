\documentclass[10pt,a4paper]{article}
\usepackage[latin1]{inputenc}
\usepackage{amsmath}
\usepackage{amsfonts}
\usepackage{amssymb}
\usepackage{framed}
\usepackage{pstricks}    %for embedding pspicture.
\usepackage{graphicx} 
% (1) choose a font that is available as T1
% for example:
\usepackage{lmodern}

% (2) specify encoding
\usepackage[T1]{fontenc}

% (3) load symbol definitions
\usepackage{textcomp}

% needed for small caption fonts
\usepackage[skip=2pt]{caption}

\DeclareCaptionFormat{myformat}{\fontsize{8}{9}\selectfont#1#2#3}
\captionsetup{format=myformat}

\begin{document}

\title{A deep-ish dive into Confidential Transactions.}
\maketitle

\section{Introduction}

What this document is not:
\begin{itemize}
\item It is \textbf{not} an explanation of sidechains.
\item It is \textbf{not} even a complete explanation of Confidential Transactions. It is already very long, but misses a few things like: how fees are dealt with, or special cases involving unblinded inputs/outputs.
\item It is \textbf{not} guaranteed to be accurate, although obviously every effort has been made. You do not have permission to put money on the line based on what is written here (although I do give full permission for you to lose all your testnet coins).
\end{itemize}

\vskip 0.1 in \noindent What this document is:
\vskip 0.1 in \noindent This serves as an amplification to \cite{ct_wu}.

\vskip 0.1 in \noindent If you want to understand in detail how CT (I will use this abbreviation for `Confidential Transactions') works, this document may be useful. How useful it is to you personally will depend on your technical level.
\vskip 0.1 in \noindent If you are a complete beginner to Bitcoin, clearly this is not for you. If you are an expert cryptographer, then the introductory sections are definitely not for you, as they include wordy descriptions of basic constructs (commitments for example). You would probably be better off going straight from \cite{ct_wu} to the code. The section at the end might serve as a useful description of the mechanics of Confidential Transactions.

\vskip 0.1 in \noindent If you are at a level where you are comfortable with the construction of Bitcoin transactions and understand what it means to add two points on an Elliptic curve, but only sketchily knowledgeable about cryptography generally, and keen to see the full picture of what Confidential Transactions is doing and how, then this might be helpful.

\subsection{Prerequisites}

What you need to understand at some level in order to fit all the following pieces together.

\begin{itemize}
\item Cryptographic hashes (here, SHA256, although that isn't important).
\item Elliptic curve arithmetic basics (what you can and can't do with elliptic curves, not *how* you do it).
\item The basic constructions of elliptic curve asymmetric cryptography: specifically, public and private keys.
\item The basic structure of a bitcoin transaction (but *not* the lower level detail, e.g. Bitcoin Script).
\end{itemize}


\subsection{Goals}

Starting in a moment, things get technical very fast. It's important to understand where we're going. The idea of having transactions be private in a digital cash context is very old. David Chaum, in particular, blazed a trail in designing a kind of money where the amount could be hidden (blinded) while it was still ensured that money was not being stolen or created from nothing. The core idea behind it is closely connected to the idea of ``homomorphic encryption''. Suppose you have a bunch of data (\$100, -\$4, \$231) and you want to give that data to someone else to compute something - like the total - but you don't want that person to know what the amounts are - either the inputs, or the output (the result). Cryptography can do this for you, and it's (in certain contexts) called ``homomorphic'' (preserving structure being the crude meaning). Some of the rules of arithmetic that apply to the plain amounts also apply to the encrypted amounts, but the encrypted amounts don't leak any information about the original amounts. This is an active area of research, generally (see "Fully Homomorphic Encryption"). To get it for arbitrary data and arbitrary computations is still between difficult and impossible. But to get it for specific calculations, especially if they're very simple (and especially if they involve \textbf{only} adding, or \textbf{only} multiplying, and not both), is to some extent a solved problem.

\vskip 0.1 in \noindent  If you're throwing your computations into the cloud for Amazon to do for you, and you're a big corporation doing god knows what, then \textit{only} being able to add numbers is a bit limiting, to say the least, and even more so if it slows the computations down appreciably. But for an application like digital cash this is not \textbf{so} much of a blockage.

\vskip 0.1 in \noindent  To give you an inkling, before diving into the details, consider this simple mathematical fact: $a^b \times a^c = a^{b+c}$. Also, the result is equal to $a^{c+b}$. Notice that the addition of b and c works, and follows the normal addition rule (``commutativity'' - you can swap them round with no effect), even \textbf{after} we've put all the numbers into exponents. This is the kind of ``homomorphism'' that we can use to get the blinding/encryption effect. In certain circumstances, a construction like that can allow you to \textbf{do arithmetic on encrypted amounts}.


\section{How to make the amounts in a transaction private}

This part is well covered in \cite{ct_wu}, but here is a brief rewriting of it for continuity. We have defined our goal as having the amount in transactions being private. So, suppose we have 3 inputs and two outputs (not atypical for Bitcoin). We spend, let's say, inputs (outputs of previous transactions) A=1btc, B=2btc, C=3btc and they are sent to outputs D=0.5btc, E=5.5btc.

\vskip 0.1 in \noindent In an ordinary bitcoin transaction, the amounts 1,2,3 and 0.5,5 would all be published into the transaction bytes, and so stored by all nodes on the network. The network verifies that 1+2+3 - (0.5+5) = 0 (ignoring fees for simplicity), and so accepts the transaction.

\vskip 0.1 in \noindent  To hide the amounts and achieve the `homomorphic encryption' mentioned in the previous section, we use a Pedersen commitment (ref !!). Cryptographic commitments are a well known powerful technique, but they are usually used in a situation where you want to promise something is true before later proving it to be true, which enables a bunch of interesting types of systems to work. 

\vskip 0.1 in \noindent  Commitments are only possible because of the existence of one-way functions; you need to be able to produce an output that doesn't, on its own, reveal the input. Cryptographic hash functions (like SHA256) perform this role perfectly. If I want to commit to the value ``2'', I can send you its hash: 
\vskip 0.1 in \noindent  53c234e5e8472b6ac51c1ae1cab3fe06fad053beb8ebfd8977b010655bfdd3c3
\vskip 0.1 in \noindent  This wouldn't be too smart though; it would mean \textit{whenever} I want to send you a commitment to the value ``2'', I would \textit{always} be sending 53....c3 - which wouldn't hide the value very well! This is a lack of what's called ``semantic security''. However, it's pretty easy to address this. 

\vskip 0.1 in \noindent Instead of committing to ``2'', I commit to ``2''+some random data, like ``2xyzabc''. To put it more generally, we decide on a format like: SHA256(secret number || 6 random characters), just for example (the real life version will be more secure of course). Because SHA256 has the property that you can't generate ``collisions'' (can't create two inputs with the same output), this still gives us the same essential security feature: once I've made the commitment, I can't change my mind on what the value is later - and you can ask me to reveal it at a later stage of a protocol/system.

So, a hash function provides the equipment to make commitments. However, what hash functions certainly don't have is any kind of homomorphism. There is certainly no rule like $\textrm{SHA256}(a) + \textrm{SHA256}(b) = \textrm{SHA256}(a+b)$. So this is no good for our purposes.

\subsection{Pedersen commitments}

It turns out you can do something very similar with exponentiation modulo $N$. First you take a generator for the group of integers modulo $N$ (ignoring the technical details on what properties $N$ must have). A generator is just a number $g$ such that 

\[g^1, g^2, g^3 .... g^{N-1} \]

produces all the integer values in the group (so it in some sense ``generates'' the whole group). Having chosen a $g$ for $N$, you can create the commitment as

\[C = g^m \ \textrm{mod}\ N \]

, where $m$ is the message you want to commit to. What's new here is what's keeping the value secret. Finding the $x$, given $C$,$g$ and $N$ is a "hard" problem - it's called the Discrete Logarithm Problem or DLP for short. It's a little like the problem of factoring very large numbers (although not the same), and like the latter, it's used to create cryptosystems with one-way functions. Input $x$, get out $C$ easily, but the reverse is essentially impossible if $N$ and $g$ are well chosen. 

\vskip 0.1 in \noindent If you do it this way, though, you have the same ``semantic security'' problem mentioned above. Pedersen's commitment system addresses this in a similar way, with a random number: $C = g^m h^r$, where $h$ is *another* generator for the same group. When you create the commitment to the value $m$ you just have to decide on a random number $r$ (the values $g$, $h$ can be set in advance; although we'll discuss this later). Then when your counterparty challenges you to prove that your $C$ corresponds to a particular $m$, you just reveal $m$ and $r$ (this is called ``opening the commitment''). The DLP mentioned above is what makes $C$ hide the value of $m$, and the use of a separate generator $h$ is what makes it impossible to create a new $m'$ and $r'$ that provides the same $C$.

You should see from this discussion that Pedersen's, as well as other, commitment scheme is providing two key properties:
\begin{itemize}
\item \textbf{Hiding} - the commitment $C$ does not reveal the value it commits to.
\item \textbf{Binding} - having made the commitment $C(m)$ to $m$, you can't change your mind and open it as a commitment to a different message $m'$.
\end{itemize}

\vskip 0.1 in \noindent This is a bit of a sidetrack, apparently, but now things come together. In Pedersen's system we can exploit the homorphism I mentioned right at the start:

\[C(A) \times C(B) = (g^A h^{r_A}) \times (g^B h^{r_B}) = g^{A+B} \times h^{r_A + r_B} = C(A+B) \]

\vskip 0.1 in \noindent It's as well to understand the mechanics of this very clearly: Alice can send $C(A)$ and, perhaps later, $C(B)$ to Bob. Bob holds those while other stuff gets done. Later, Bob can treat the product of those two values as a commitment to the sum $A+B$. In doing so, he would never know what $A$ and $B$ originally were, but he could be sure that mathematically $C(A+B)$ \textbf{must} be a commitment to the sum of A and B. Notice that Alice could have sent $C(A+1)$ and $C(B-1)$ and it would make no difference to what Bob sees.

\vskip 0.1 in \noindent What about the random $r$ values? Notice that in $C(A+B)$, the $r$ value that Alice gives, if she opens the commitment (reveals $A+B$), must be the sum of $r_A$ and $r_B$, otherwise the final value will be wrong. But as long as she takes care to make sure that sum holds, it doesn't actually matter which $r$ values she chooses. We'll see this pattern play out later.

\subsection{Moving to the Elliptic Curve version}

We can do all of this with elliptic curve points as our basic mathematical objects, instead of numbers. The gory details (e.g. how to add two elliptic curve points together) need not concern us, only that elliptic curve point addition is analogous to multiplication in the above system. Thus, when we look at:
\[C = mG \quad \textrm{ - elliptic curve point G added to itself m times} \]
it performs pretty much exactly as:
\[C = g^m \quad \textrm{ - number g multiplied by itself m times} \]
did before. $G$ is now a generator \textbf{point} (sometimes called base point). Just as before, it is hard (effectively impossible) to derive $m$ from $C$ and $G$. $C$ is now a point rather than a number, so has coordinates on the curve $(x_C, y_C)$. This is exactly the same kind of crypto security as protects your bitcoin balances. It is called the ECDLP, just like the DLP but for elliptic curves. Obviously the question of how secure it is is a huge and highly technical topic. Bitcoin, and CT, works on the assumption that it is "computationally secure" - that is, it could theoretically be solved/cracked, but the computation cost of doing so is so large that in practice it's impossible.

\vskip 0.1 in \noindent (A technical point, but one as well to get out of the way now: To represent such a point as a number (or a string of bytes), we don't have to print out the values of $x$ and $y$ (both of which are 32 bytes or 256 bits), but just the $x$ value plus a single flag to indicate positive or negative  $y$. That's because the curve is defined by an equation of the form $y^2 = x^3 + ax + b$, so for every positive $y$ value solution, there is one negative $y$ value solution. So an EC point (for a 256 bit curve, anyway) can be encoded into ~33 bytes.)

\vskip 0.1 in \noindent So, it's immediately clear what a Pedersen commitment would look like in the Elliptic Curve context. Instead of 
\[C = g^m \times h^r \]
we get:
\[C = mG + rH \]
This still keeps the same homomorphic property, although we add the commitments rather than multiplying them:
\[C(m_1) + C(m_2) = m_{1}G + r_{1}H + m_{2}G +r_{2}H = (m_1+m_2)G +(r_1 + r_2)H = C(m_1+m_2) \]

However, for technical reasons we change the notation here. Instead of the message being $m$ and the random number being $r$, we will use $a$ for the amount (because the message we're going to be committing (and hiding) is the bitcoin amount) and $x$ for the random number, which we're going to call a `blinding factor' (remember `semantic security' - we don't want someone to be able to figure out what the $a$ is just by looking at the $C$, so the random $x$ value `blinds' it). So now we have:
\[C = xG + aH\]
A note about the generators: $G$ is the generator that's specified as part of the parameters of the secp256k1 curve that Bitcoin uses; i.e. it's exactly the same G as in:

\vskip 0.1 in \noindent Public key = (private key) * G

\vskip 0.1 in \noindent $H$, on the other hand, is not part of the secp256k1 specification. We need $H$ to be basically a random number. But, just throwing a random one in there would not be secure (at least, not secure against the person who threw it in!) Why? Suppose an evil system designer chooses $H$, such that he knows $x_H$ obeys:
\[H = x_H G \]
This means he can change the amount $a$ that the commitment $C$ commits to, to anything he likes. In other words, we lose the property of `binding' described above. Here's how it would work:
\vskip 0.1 in \noindent The attacker constructs a commitment to the value $a=2$: he first makes the EC point $2H$, then chooses a random blinding factor $x$, constructs $xG$ and adds it in: $C(2) = xG + 2H$. Later, he decides he would prefer the value to be $a=3$. If he doesn't know the value $x_H$ then he's stuck: he wants $C(2)$ to be $x'G + 3H$, which leaves him trying to find the value $x'$ that makes that work. But all he has is a new EC point: $C(2)-3H = x'G$ and from this he \textbf{cannot} find $x'$ - that's exactly the ECDLP, that is supposed to be unsolvable by the system design.
\vskip 0.1 in \noindent However, our attacker has extra information: he knows $x_H$ such that $H = x_H G$. For him, the equation for $C(2)$ looks like this: 
\[C(2) = xG + ax_{H}G = \left(x+2x_{H}\right)G \]
If he wants to make $C(2)$ a commitment to 3 instead of 2, he just has to do simple arithmetic with the scalar numbers in parentheses: 
\[C(2) = \left(\left(x-1x_{H}\right) + 3x_{H}\right)G \] 
So he just publishes $x' = x-x_H$ as his new blinding factor, and $a=3$, and he's made a $C(3)$ which is the same as his earlier claimed $C(2)$.

\vskip 0.1 in \noindent The upshot is, that in order for the Pedersen commitment scheme to work (without trust), we need $G$ and $H$ to be what is sometimes called `NUMS' - nothing up my sleeve numbers. There have been considerable discussions about the NUMS status of $G$, the generator for secp256k1 as used in Bitcoin ( ref!!). Assuming that it is indeed a NUMS (or, just trustworthy let's say), we are left here with the need for $H$ to also be a NUMS.

\vskip 0.1 in \noindent For this reason, in CT, $H$ is chosen in a special way. The value of $H$ is the SHA256 hash of a simple encoding of the pre-specified $G$ (specifically, its $x$-coordinate). SHA256 outputs are random by design, so unless one believes that the creators of secp256k1 deliberately (and how?) found a $G$ such that if someone in the future hashed it and used it as a generator for a Pedersen commitment, it would create a value that they knew the discrete log of, we can trust that no one knows a $x_H$ such that $H = x_H  G$.

\vskip 0.1 in \noindent (Amusing side note: the pubkey $H$ in Bitcoin corresponds to the address:
\vskip 0.1 in \noindent 1D8eDztgv79J59V7UBBpNGnRE6hjstqKb5
\vskip 0.1 in \noindent If anyone ever published a spend from that address on the Bitcoin network, it would imply that $x_H$ is known, which would imply that the system described here is broken!)

\subsection{The basic layout of a blinded transaction}

The above section goes into a lot of detail about how Pedersen commitments work, because to really understand the security of Confidential Transactions, this is critical (and it will be developed further). So what does a transaction look like - here's an example just to frame the discussion.

\vskip 0.1 in \noindent A normal Bitcoin transaction in rudimentary form:
\begin{verbatim}
Inputs            Outputs
A = 1 BTC
                  D = 0.5 BTC
B = 2 BTC
                  E = 5 BTC
C = 3 BTC
\end{verbatim}

Fee ignored for simplicity - so the amounts obey $A + B + C - (D + E) = 0$. The amounts are stored as 64 bit integers, taking up 8 bytes each in the transaction as stored on the blockchain. To blind the amounts, we are going to replace those 8 byte amounts with a Pedersen commitment to each of the amounts. These will obey $C(A) + C(B) + C(C) - (C(D) + C(E)) = 0$. First, note that this means replacing each 8 byte amount with a 33 byte commitment (it was explained in the previous section why these Pedersen commitments should be 33 bytes). Second, remember from:
\[C(m_1) + C(m_2) = m_{1}G + r_{1}H + m_{2}G +r_{2}H = (m_1+m_2)G +(r_1 + r_2)H = C(m_1+m_2) \]
... that if we want $C(A+B)$ to be zero, it needs not only that $A+B$ is zero, but also that $r_A + r_B$ is zero. So it's necessary for the blinding factors chosen in constructing all the commitments also adds up to zero. We could draw this out something like this:

\vskip 0.1 in \noindent A blinded-amounts Bitcoin transaction in rudimentary form:
\begin{verbatim}
Inputs            Outputs
A = C(1,x_A)
                  D = C(0.5,x_D)
B = C(2,x_B)
                  E = C(5,x_A + x_B + x_C - x_D)
C = C(3,x_C)
\end{verbatim}

where $x_{A,B,C,D}$ are random values chosen in transaction creation (32 byte values). The blinding factor for the last number (although it could be any of the 5) is chosen to ensure that:
\[C(1,x_A) + C(2,x_B) + C(3,x_C) - C(0.5,x_D) - C(5,x_A + x_B + x_C - x_D) = 0\]
- because the blinding factors add up to zero.

\vskip 0.1 in \noindent It's important to take a step back at this point and see how this looks from different perspectives - that of the Alice, the sender, Bob, the receiver, and most importantly, Nelly, the node on the sidechain/Bitcoin network who relays this transaction or mines the block. Alice is the one that constructs all this data. For her inputs, she is taking the commitments (including blinding factors) from a previous output, so this data is set in advance. She constructs output amounts as she chooses, taking care that all the blinding factors together add to zero as mentioned above, and that the amounts (ignoring fees) also add to zero, then she can make output commitment values that will obey the property that their sum is zero.

\vskip 0.1 in \noindent Now the transaction is published to the network and received by Nelly. Nelly can simply check that $C_A + C_B + C_C - C_D - C_E = 0$. She has no idea about blinding factors, and doesn't need to (remember the key properties of commitments? hiding and blinding). Again, ignoring fees, Nelly can mark the transaction as valid and propagate it (or mine it), because just that fact of the sum being zero means that no new Bitcoins were illegally created. The newly created unspent transaction outputs (``utxos''), with their commitments, represent valid tokens to be spent in future on the network.

\vskip 0.1 in \noindent The pretty picture painted in the preceding paragraphs about how the transaction can be valid but its amounts kept secret - actually has a gaping security hole in it. If you've already read (ref!!), you'll know what I mean. If not, consider it a puzzle - what's wrong with this picture? I'll give you a clue: all calculations in the elliptic curve group secp256k1 are modulo $N$, where $N$ is the size of the group.

\vskip 0.1 in \noindent If you can't figure it out, no worries - the explanation of this problem and its solution form the last, and largest section of this document. So back to that in a bit.

\vskip 0.1 in \noindent The receiver Bob now has these new utxos, for D and E (let's say he owns both of the corresponding addresses). But here we hit a snag - he doesn't know the amounts! So he's grateful to Alice, but not so grateful - it's difficult to spend coins when you have no idea what their denomination is. If that wasn't bad enough, he also doesn't know the blinding factors, so he can't construct a new transaction at all.

\subsection{Enter ECDH}

To solve the problem mentioned at the end of the last section, Alice could send an encrypted message to Bob with the amount and blinding factor for the commitments he receives (like, in our case of $C(0.5,x_D)$, send him the number '0.5' and the value of $x_D$). Using off-network communication like encrypted email is obviously a terrible idea, not least because it requires knowledge in advance of the receiver's identity, thus breaking Bitcoin's pseudonymous model.

\vskip 0.1 in \noindent A thought that might occur to you is this - since Bitcoin is based on public/private key pairs, couldn't we just have the sender encrypt the necessary information using the receiver's public key, and send the encrypted data with the transaction? Unfortunately, no - the receiver, in Bitcoin, publishes an \textbf{address}, which is a hash of a public key, and not the public key itself. So again it would require coordination with the receiver in advance - ``give me your public key please'', and it's at least theoretically less secure to publish raw public keys instead of addresses. Even if this model \textbf{did} work, it would have the bad side effect of requiring a bunch of extra data to be added to the transaction, and that is highly undesirable.

\vskip 0.1 in \noindent What's needed is a way to share a secret between the sender and receiver without the receiver's involvement, and moreover to then find a space-saving way of embedding the secret information into the transaction without leaking it to Nelly or anyone else traversing the blockchain.

\vskip 0.1 in \noindent Diffie Hellman (ref!!), or DH, key exchange is a standard way to solve that problem. Your browser may well be using it right now (e.g. on Firefox go to about:config, type `ssl3' into the search bar and you will see strings like `security.ssl3.dhe\_rsa\_aes\_128\_sha' - here the `dh' part refers to Diffie-Hellman key exchange). The basic idea is not a lot different from our earlier discussion of (specifically Pedersen-style) commitments. We will immediately jump to the elliptic curve form (ECDH), although it works the same with discrete log form.

\begin{itemize}
\item Alice has public key $A = aG$, so private key $a$.
\item Bob has public key $B = bG$.
\item Alice and Bob exchange public keys.
\item The shared secret is just $S = aB = bA$. 
\end{itemize}

Alice can calculate the former ($aB$), Bob can calculate the latter ($bA$). Notation is tricky here; you might immediately think ``anyone can find that secret, it's just $A \times B$'', but no: there is no meaning to $A \times B$; you cannot multiply elliptic curve points together. You can only multiply an EC point by a scalar (a number) (and of course `multiply by a scalar' just means add the point to itself that many times). No one not knowing one of $a$ or $b$ can calculate $S$.

\vskip 0.1 in \noindent CT uses ECDH to achieve a powerful result: not only can the sender transfer the value and blinding factor information in encrypted form (encrypted in a special sense, but functionally it's encryption), but she can do it without using \textbf{any} more space in the transaction, and crucially \textbf{without any interaction in advance by the receiver}. This last point is why the previous suggestions were not even worth considering - in an online payment system like Bitcoin it's essential that the receiver does not have to ``be there''.

\vskip 0.1 in \noindent We're going to see in the next section how exactly this works.

\vskip 0.1 in \noindent To enable this, the parties to the transactions have to have separate public keys specifically for this ECDH purpose. They are called ``scanning keys'', since they enable scanning/reading of the private transaction information. Since they have to be public, they are included in the new format for addresses. Here is an example from the current Elements Alpha testnet:

\vskip 0.1 in \noindent 22E8QKHaTijFemPDwKvAk9qoTgagPfp8nBQiry87MMU1h2gQTeYjhAKbHWHcTZ5N6hmHpmdLuoVsYZb9e

\vskip 0.1 in \noindent You'll notice it's a lot longer than a Bitcoin address; it is actually constructed as:

\vskip 0.1 in \noindent base58check(<scanning key version byte> <normal address version byte> <scanning key> <normal address data bytes>)

\vskip 0.1 in \noindent Base58 check is the special encoding Bitcoin uses to format addresses, and uses a checksum to prevent errors (see ref!!).

\section{The range proof}

\subsection{The problem}

Earlier I set the puzzle: why is a construction like this:

\vskip 0.1 in \noindent A blinded-amounts Bitcoin transaction in rudimentary form:
\begin{verbatim}
Inputs           Outputs
A = C(1,x_A)
                 D = C(0.5,x_D)
B = C(2,x_B)
                 E = C(5,x_A + x_B + x_C - x_D)
C = C(3,x_C)
\end{verbatim}
and 

\[C(1,x_A) + C(2,x_B) + C(3,x_C) - C(0.5,x_D) - C(5,x_A + x_B + x_C - x_D) = 0\]

\vskip 0.1 in \noindent ..unsafe? The answer lies in the nature of modular arithmetic. Let's make a toy example: suppose we take the group of integers modulo 11. We can add 3 and 4 and get 7; we can use our new fancy crypto system for commitments, and know that $C(3) + C(4)$ will equal $C(7)$, even though someone looking can't tell what values the commitments correspond to. Great. But what happens when we do $C(9) + C(5)$? The problem is that $9 + 5 = 3 \ \textrm{mod}\ 11$. This output of `3' has been created in an illegal way, and although that precise case is not harmful (the output is too small), it makes for a very easy exploit. An example:

\vskip 0.1 in \noindent How to construct illegal but undetectable free coins (toy system with modulus N=11):
\begin{verbatim}
Inputs           Outputs
A = C(1,x_A)
                 D = C(9,x_D)
B = C(1,x_B)
                 E = C(4,x_A + x_B - x_D)
\end{verbatim}


\vskip 0.1 in \noindent Now let's consider how the commitments look to Nelly the node. Because $1 + 1 -9 - 4 = -11 = 0 \ \textrm{mod} \ 11$, then adding the input commitments and subtracting the output commitments yields zero. Nelly says - OK! Meanwhile our attacker Alice has created lots of free coins: 13. In reality, the `9' amount might be illegal if we changed to the real $N$ (a 32 byte number, quite a lot bigger than 11!), since it might be bigger than the valid range (even if $< N$), but even if that is the case - the other output of `4', being small, could be be created freely as illegal coins. The network would never know this and the attacker could spend \textit{this} outputs in new transactions.

\vskip 0.1 in \noindent So modular arithmetic, an essential part of the cryptography, scuppers, it seems, the grand plan to hide the transaction amounts with Pedersen commitments. It would work fine with ordinary numbers in the set of integers (the infinite set), but not in a finite group of integers, modulo $N$ for some $N$ such as that in secp256k1.

This is the final problem that we need to solve in order to make Confidential Transactions work, and its solution is the most sophisticated part.

\subsection{Borromean Ring Signatures}

A ring signature solves a problem that's pretty easy to understand: it's related to the famous ``Dining Cryptographers'' problem \cite{chaum_dc}, and the idea is expanded upon in Section 2.2 of \cite{ms_ringsig}. We want to know that one thing is true (in that case, that one person attests to the statement/leak), but we specifically \textbf{don't} want to know \textbf{which} one. It probably isn't at all obvious, but that kind of system could solve the modular arithmetic overflow problem above. Let's reframe the problem. Instead of just knowing that:

\[A + B + C - D -E \]

is true, we also want to know:

\[0 <= A < 100\]

for example, and the same for all of $B$,$C$,$D$,$E$. If $N$ is a huge number and all the numbers in our transaction are less than 100, then we're OK: you can't add 3 numbers less than 100 and get anything more than 300, so overflow of the type we discussed above would be impossible.

\vskip 0.1 in \noindent That means, what we're looking for is a situation where the transaction creator (Alice from earlier) can only create a valid commitment such as $C(A)$ \textbf{if} $0 <= A < 100$. Now we can't do that directly - limit the value that Alice commits to - but what we can do is \textit{add} an extra piece of data, which we will call a ``rangeproof'', that proves that the commitment $C(A)$ is a commitment to a value in the range $(0,99)$. It's as if we had 100 cryptographers at the dining table, each one having names $0,1,2..99$, and we could take a ring signature from the group, knowing that 1 of the 100 made the signature; the extra detail is that \textit{he could only sign our special signature because his name was equal to the value of A}. 

\vskip 0.1 in \noindent So we know that $A$ is in that range 0 - 99, but we don't know what it is. If we then attach that special signature (=rangeproof) to our input $A$ in our transaction, and do the same for $B$, $C$, $D$, $E$, then we have achieved the goal - the transaction, including the rangeproof for each input and output, \textbf{must} be valid, even though we don't know the amounts. Very important to see: what guarantees the ranges is that Alice couldn't have created valid rangeproofs unless $A$ (and $B$,$C$,$D$,$E$) is in that specific range we decided on.

\vskip 0.1 in \noindent Now, the basic idea of a ring signature has been known for some time, and there is certainly more than one way to create it. The Borromean ring signature \cite{borromean} is a carefully optimised form of a signature based on a clever combination of hash functions and elliptic curve calculations (the latter very similar to what we've already discussed for Pedersen commitments):

\vskip 0.1 in \noindent Each "participant" (a cryptographer at the dining table, or a number in the range) has their associated public key, $P$. Let's called the first such participant Boris with public key $P_B$ (whose corresponding private key is $x_B$), and trace through what he can (and can't!) do to make a signature.

\vskip 0.1 in \noindent He starts by making up a random number $k$, called the 'nonce' (common nomenclature in cryptography, meaning `number used once'). He then makes the corresponding EC point $K = kG$ (we are still working with Bitcoin's curve secp256k1 and its corresponding $G$, let's say). He takes the message he wants to sign, let's call it $m$, and constructs the hash $H(K||m)$ where $K$ has been put into a 33 byte form as usual, and $m$ has been appended to it, and we call this hash value $e$. Then, he finds the signature as follows: $s = k + x_{B}e$ (scalar numbers, not EC points). Finally, he publishes his signature as $(s,e,m)$, tied to Boris' pubkey $P_B$.

\vskip 0.1 in \noindent Now, suppose you want to verify the signature as valid. You take $s$ and multiply by $G$: $sG$, and subtract $eP_B$. This gives you:

\[sG - eP_B = sG - ex_{B}G = (s - ex_{B})G = kG = K \]

So now you have $K$ (although note! you never had $k$, nor $x_B$, and according to ECDLP, you can't work them out).
Then, you just reconstruct the hash: $H(K || m)$. Does it equal $e$? If so, the signature is valid.
\vskip 0.1 in \noindent The thing to notice about this is the weird feedback loop when verifying: we needed to get $e$ to make the signature, to check it, but in this verify process, we take $e$ as one of the inputs $(s,e,m)$ and then produce it as the final output. Of course, there is no feedback loop in the initial signing - you can't use the output of the function as your input!

\vskip 0.1 in \noindent Just as important as knowing how to verify, is to know why it's impossible to forge the signature if you're not Boris. 

\vskip 0.1 in \noindent Suppose you don't know the private key $x_B$. Then you can't work out $s$ for your signature to work. You can work out $S = sG$, because you can add $kG + eP_B$, but the scheme specifically requires you to publish the scalar number $s$.

\vskip 0.1 in \noindent So far, we have just created a signature, not a ring signature. The genius idea here is to \textbf{chain the signatures together}. (I strongly recommend reading the Borromean paper \cite{borromean} if you want a detailed, clear explanation for this; but I will do my best to create a somewhat dumbed down version here).

\vskip 0.1 in \noindent Say the participants are Boris, Charlie, Donald and Edward. Form a loop/ring: $B \rightarrow C \rightarrow D \rightarrow E$ with $E$ connecting back to $B$. What we're going to do is to make a signature, just as above, for all 4, but with 2 differences: (1) all except one of the signatures will be forged/invalid, (2) each hash value $e = H(K||m)$ be a hash of the $K$ value for the public key \textit{before} it in the loop. So $C$'s $e_C = H(K_B || m)$, $e_D = H(K_C||m)$ etc.

\vskip 0.1 in \noindent This structure allows *one* of Bob, Charlie, Donald and Edward to make a signature - the process is as follows. In this remember that the public (verification keys) for each participant, $P_B$, $P_C$, $P_D$, $P_E$ are known by everyone, while the corresponding private keys ($x_B$ etc.) are known only by that party.

\vskip 0.1 in \noindent We'll assume Donald is the one to make the signature.
He's going to construct a ring (loop) of signatures; he can start at any point. 
This is one way he could do it:

\begin{enumerate}
\item Choose a nonce for himself, $k_D$, at random. Make $K_D = k_{D}G$.
\item Make the message hash for the next node Edward: $e_E = H(K_D||m||E)$ (we append 'E', or a number, to reference this vertex in the ring).
\item Choose a signature value $s_E$, for Edward, at random. From this, calculate the corresponding $K_E$ value: $K_E = s_{E}G +e_{E}P_{E}$.
\item Calculate the e-value for Boris, the next in the loop: $e_B = H(K_E || m || B)$.
\item Choose a signature value $s_B$ for Boris, at random, and calculate $K_B = s_{B}G + e_{B}P_{B}$.
\item Calculate the e-value for Charlie: $e_C = H(K_B || m || C)$.
\item Choose the signature at random for Charlie: $s_C$. $K_C = s_{C}G + e_{C}P_{C}$.
\item Calculate the e-value for his own node in the ring: $e_D = H(K_C || m || D)$.
\item Finally, he can make a \textbf{real} signature $s_D$ for himself: He needs $s_D$ such that $K_D$ (we chose it in step 1) $= s_{D}G + e_{D}P_{D}$. To do that, we rearranges the equation:
\begin{eqnarray*}
K_D &=& s_{D}G + e_{D}P_D \\
k_{D}G &=& s_{D}G + e_{D}x_{D}G \\
s_{D}G &=& \left(e_{D}x_{D} - k_{D}\right)G \\
s_D &=& e_{D}x_{D} - k_D
\end{eqnarray*}
\end{enumerate}

\vskip 0.1 in \noindent ... and it can be seen, as in previous explanations, that his ability to do this is contingent on his knowledge of the private key $x_D$. After calculating this he can publish the complete set of signatures:
\[ (e_B, s_B, e_C, s_C, e_D, s_D, e_E, s_E) \]
\vskip 0.1 in \noindent Actually, all the e-values are determined if one is published along with all the s-values, so this is fine:
\[(e_B, s_B, s_C, s_D, s_E)\]
\vskip 0.1 in \noindent So to publish a ring signature for 4 parties in this system, we need to publish 5 32 byte numbers. 3 of the 4 s-values are just random nonsense, but the actual, real signature guarantees that a verifier will know that one of the 4 has signed with his private key, while not being able to tell which (as the numbers are indistinguishable from random).

\vskip 0.1 in \noindent How does it look from a verifier's point of view? He can start at the node on the ring for which he has the e-value (it can be any, but in this case Boris). He can calculate $K_B = s_{B}G + e_{B}P_B$. From that, he can calculate the next e-value: $e_C = H(K || m || C)$, check the next sig value, and so on round the loop until he comes back to $e_B$ at the beginning. The signature is verified if he can recreate this hash correctly.

\vskip 0.1 in \noindent A high level understanding might be: a single signature uses a feedback loop to prevent forging; the ring signature version stretches out the feedback loop over a set of participants by using the chaining-signature design. So the signature is still impossible to forge in toto, unless you can short-circuit that feedback by having one or more of the private keys.

\vskip 0.1 in \noindent What is described above is the most important part, but it is basically the "AOS" signature design as referred to in \cite{borromean} section 2.2. The Borromean structure adds one extra feature, to boost the compactness of the structure. Take two loops like those above and "join" one of the nodes. For example, suppose along with a signature over (Boris, Charlie, Donald, Edward) we also wanted a signature over (Bertha, Carol, Diana, Elizabeth). We keep the same structure as above for the two rings, except we ``join'' Boris and Bertha into one node. They still calculate their hashes like $e_B = H(K_E || M || B)$ but then we make a hash of those two hashes together: $e_B = H(H(K_{Edward} || M || Boris) || H(K_{Elizabeth} || M || Bertha))$. The effect of this is to make a single signature for which the loops cannot be validated without their being at least one genuine signer in \textbf{both} loops. Logically, this means we get a signature which can only be verified if (Boris or Charlie or Donald or Edward) AND (Bertha or Carol or Diana or Elizabeth) signed.

\vskip 0.1 in \noindent You can join more than 2 loops at that one node (the 'B' node above). You can see that it doesn't actually matter \textit{which} node you join at, so for simplicity one can have $N$ loops joining all at the same node index.

\vskip 0.1 in \noindent What does the joining give us? A significant space saving in publishing the signatures. For a trivial example like the one above, with 2 loops joined together, it makes almost no difference - 2 loops separately would need 8 nodes and so a sig like ($e$, s(Boris), ....s(Edward),$e$,s(Bertha),... s(Elizabeth)), which you can see would be 8 + 2 = 10 numbers, all 32 bytes. With the Borromean style (connected), you'd need one less $e$ value (same $e$ value for both), so 9 numbers.
\vskip 0.1 in \noindent But now imagine a case of 32 rings and 4 in each. Then without the Borromean topology, you'd need 32 * 4 + 32, whereas with it you'd need 32*4 + 1, which is a significant space saving (approaching 20\%).

\vskip 0.1 in \noindent Whether other similar structures might offer more space savings or other advantages seems to be an open question according to \cite{borromean}.

\subsection{Implementation of a range proof}

We now have the conceptual tools in place to solve the problem ``how to prove that a number is in a range'', but the devil is as always in the details. What actually happens in the code? We will try to figure it out concretely, using numbers and code snippets where appropriate. For the latter, I'm using Python scripting with pybitcointools \cite{pbtc} as my `backend' for elliptic curve calculations in secp256k1.

\vskip 0.1 in \noindent Let's imagine Alice wants to create an output in her transaction, of size 1 BTC. The code will have to define what an appropriate range is: minimum 0, maximum 4294967295 satoshis, or 42.94967295 BTC. Why this particular number? Because it's $2^{32}-1$, so it's the maximum value we can store in a 32 bit integer data type. Remember that the idea is to create a Pedersen commitment to this value, $C$, and an associated range proof that its value is between 0 and 4294967295 satoshis. Its actual value will be 100000000 satoshis, but the network won't know that. We proceed as follows:
\begin{verbatim}
	>>> import bitcoin as btc
	>>> import os
\end{verbatim}
We have $G$, the generator for the curve built in, but we'll also need the secondary generator $H$ as described in the ``Pedersen commitments'' section. Let's find its value, using the construct: $x$-coordinate of $H$ is the integer value of the sha256 digest of the uncompressed encoding of the generator $G$:
\begin{verbatim}
	H_x = int(sha256(btc.encode_pubkey(btc.G,'hex').decode('hex')).hexdigest(),16)
	>>> H_x
	36444060476547731421425013472121489344383018981262552973668657287772036414144L 
\end{verbatim}

We use a known mathematical trick to find $H_y$ efficiently (see ref!!):
\begin{verbatim}
	>>> H_y = pow(int(H_x*H_x*H_x + 7), int((btc.P+1)//4), int(btc.P))
	>>> H_y
	93254584761608041185240733468443117438813272608612929589951789286136240436011L
	>>> H = (H_x, H_y)
\end{verbatim}

Just for completeness, here's the compressed format for $H$:
\begin{verbatim}
	>>> btc.encode_pubkey(H,'hex_compressed')
	'0350929b74c1a04954b78b4b6035e97a5e078a5a0f28ec96d547bfee9ace803ac0'
\end{verbatim}

Let's build the Pedersen commitment for our amount of 100000000 satoshis. Remember that the construct is:
\[C = xG + aH \]
where $x$ is our blinding factor for this output. We can start by choosing that randomly:
\begin{verbatim}
	>>> x = os.urandom(32)
	>>> import binascii
	>>> binascii.hexlify(x)
	'c423ee7e2758e86254e110fd39eee4eaa232cf94fbaab710fdf811586937499e'
\end{verbatim}

This blinding factor will be kept secret between ourselves and the receiver of the output. Next, let's calculate the commitment $C$:
\begin{verbatim}
	>>> C = btc.fast_add(btc.fast_multiply(btc.G, btc.decode(x,256)),btc.fast_multiply(H,100000000))
	>>> C
	(33716128218899052265976399347815908056340330488994131284759589637119207433518L,
	52864758971156851515686729448875905407492270100334584954397099441012230985012L)
\end{verbatim}

Let's confirm that the Pedersen commitment scheme works. We'll make two other commitments $C2$ and $C3$, which sum to the same amount, and whose blinding factors sum to $x$, and show that $C - C2 - C3 = 0$.
\begin{verbatim}
	>>> C2 = btc.fast_add(btc.fast_multiply(btc.G, 7),btc.fast_multiply(H,50000000))
	>>> C3 = btc.fast_add(btc.fast_multiply(btc.G, btc.decode(x,256)-7),btc.fast_multiply(H,50000000))
	>>> sum = btc.fast_add(C2,C3)
	>>> sum
	(33716128218899052265976399347815908056340330488994131284759589637119207433518L,
	52864758971156851515686729448875905407492270100334584954397099441012230985012L)
\end{verbatim}

As you can see, the result $= C$. I chose 7 and $x-7$ as two random numbers that added up to $x$. This is the basic integrity promise of the system - amounts are hidden but sums are preserved. Now, with our blinding factor $x$ in hand, we can construct a range proof - a proof that our amount was in the range 0 to 4294967295 satoshis.

\vskip 0.1 in \noindent The decimal amount 100000000 in binary is: 00000101111101011110000100000000. The leading zero digits can't be cut off, because remember the point of the exercise is to prove that it's \textbf{some} number in the range up to $2^{32} -1 = 4294967295$, without giving away which. If we want to construct a ring signature over each of these digits, we'd need 1 ring per digit and 2 pubkeys/verification keys per ring. To publish the signature, we'd need to publish the $e$ value, all the $s$ values (see the section on the Borromean ring signatures), and also the pubkeys we chose for each of the rings. So the total number of values we'd need to publish is: 1 + 32*2 + 32, or 97. This is an oversimplification, partly because the pubkeys are one byte longer than the signature and e values (33 instead of 32 bytes), and for one other technical reason that'll come up later. But it's good enough for now. Now 97*32 is about 3KB, but we can do better than this in terms of space usage.

\vskip 0.1 in \noindent Suppose we use a different base for the number. Using non-powers of 2 is messier in programming terms, so at least we can consider base 2,4,8 etc. In these cases, the total number of (pubkeys + signatures) we need for these 32 bit numbers is : $32*(1+2)$, $16*(1+4)$, $11*(1+8)$ (11 because: if base 8, that's 3 binary bits per digit, and the last digit will have 2 bits because 3 doesn't divide 32 exactly. 8 because in base 8, there are 8 possible values for each digit, and we need one signature for each). As a formula for any number of bits-per-digit, this is: total size of pubkeys + s-values $= \left(\frac{32}{x}\right) \times \left(1 + 2^x\right)$. This function takes its minimum value at around $x=2$, which means 2 bits per digit. (As noted before, there is some oversimplification here, but nothing that changes the result - 2 bits per digit is optimal).

\vskip 0.1 in \noindent 2 bits per digit means base 4, and that's what CT uses for its representation of amounts. This means we need $16*5 = 80$ 32-byte binary strings to encode the Borromean ring signature, or 2560 bytes. Actually, this isn't exactly accurate, as we shall see in the following details, but it's a good approximation.

\vskip 0.1 in \noindent So given this, let's continue. First, rewrite the amount in base 4:

\[ 0011331132010000 \]

\vskip 0.1 in \noindent So we'll have 16 rings in the signature, each with 4 pubkeys corresponding to digit values (0,1,2,3). Constructing the Borromean signature will require us, as discussed in the section on that topic, to arrange for one of the pubkeys for each of those digits to be 'signable-for'. Let's take the third digit, which should have value '1', as an example. That amount is $4^{13} = 67108864$. So we need to construct a pubkey like this:

\[C_{2} = x_{2}G + 67108864H \]

\vskip 0.1 in \noindent The index is 2 and 2 because it's position 2 in the digits (3rd ring, starting from zero). So we need a secret blinding factor $x_{2}$, and corresponding blinding factors $x_{i}$ for all 16 rings. Once we have that, we can construct a set of 4 pubkeys for that digit $C_{20}$, $C_{21}$, $C_{22}$, $C_{23}$ where the second index is over the possible digit values. We do it like this:

\begin{eqnarray*}
C_{20} &=& C_{2} - 0H \\
C_{21} &=& C_{2} - 67108864H \\
C_{22} &=& C_{2} - 134217728H \\
C_{23} &=& C_{2} - 201326592H
\end{eqnarray*}

Now, when time comes to do the Borromean sign, we will be able to construct a \textbf{real} signature for $C_{21} = x_{2}G + 67108864H - 67108864H = x_{2}G$, because we have its private key: $x_{2}$, but not for any of the other values. Although the network will only know we signed one of the 4, not which one, it will still be satisfied with our arithmetic for the whole transaction, because it will check that this holds:

\[C = C_{0} + C_{1} + C_{2} + \ldots + C_{15}\]

where $C$ is the commitment for this output, namely the pubkey we calculated before:

\begin{verbatim}
(33716128218899052265976399347815908056340330488994131284759589637119207433518L, 52864758971156851515686729448875905407492270100334584954397099441012230985012L)
\end{verbatim}

To do this, we have to make sure that the blinding factors $x_{0} + x_{1} + x_{2} + \ldots + x{15}$ add up to $x$, i.e. the value (in hex) c423ee7e2758e86254e110fd39eee4eaa232cf94fbaab710fdf811586937499e, that we calculated as the overall blinding factor for $C$.

\vskip 0.1 in \noindent So far, so good, but let's not forget - the receiver of the transaction needs to be able to check that the claimed amount is correct. How is that achieved?

\subsection{RFC6979 deterministic signatures}

RFC6979 \cite{rfc6979} is a procedure for generating pseudo random numbers to use in ECDSA signatures; its motivation is to make such signatures safer, because a failure to use random numbers in a key part of the ECDSA algorithm can lead to catastrophic failure (exposing private keys). 

\vskip 0.1 in \noindent The pseudo random numbers are created in a kind of standard cryptographic way - you start with a seed value, and then uses hashes to create more random values, as many as you like. The procedure to do that, in this case, is something called HMAC \cite{hmac}. The simple version is that HMAC takes a specific hash function (in this case SHA256) and a `key' - a secret random number - and then generates a hash output that depends on that random number. You can then repeat this process, taking each output as the next input to the HMAC process, as many times as necessary. We call the very first input the `seed'.

\vskip 0.1 in \noindent It's called deterministic because it always creates the \textbf{same} stream of pseudo-random numbers given the same seed. This is useful in exactly the context of CT - we want a big set of random numbers (the blinding factors) that look totally random to an outsider, but which will be the same for any 2 parties who know the secret value used to seed the process.

\vskip 0.1 in \noindent You'll remember from earlier that CT addresses are longer than normal Bitcoin addresses, like this:

\vskip 0.1 in \noindent 22E8QKHaTijFemPDwKvAk9qoTgagPfp8nBQiry87MMU1h2gQTeYjhAKbHWHcTZ5N6hmHpmdLuoVsYZb9e

\vskip 0.1 in \noindent and that embedded in this is a ``scanning key'' for doing ECDH. Specifically this 32 byte key is used to seed the RFC6979 HMAC procedure, so the receiver as well as the sender can generate the same blinding factors $x_{0}$, $x_{1}$, $x_{2}$, ... $x{15}$ (except the last, $x_{15}$, is not random, but chosen so that the total is $x$, as previously mentioned).

\vskip 0.1 in \noindent However, CT uses this procedure to do more than just generate shared secret blinding factors. You'll recall that a Borromean ring signature uses a lot of random numbers - basically, to forge the majority of the signatures of the individual pubkeys. CT uses this to its advantage - it keeps using the HMAC to generate random numbers for every one of the signatures in the Borromean ring sig. Specifically, it calls a function rfc6979\_hmac\_sha256\_generate() to create 32 pseudo-random bytes for every $s$ value in the set of 16*4 in the ring sig. An important technical note: since $\frac{1}{4}$ of these signatures will have to be real, rather than forged, the random values for the real signatures are then moved into the $k$ values for the real signatures (what was called the `nonce' for the signature in the section on Borromean above, see $k_D$ in step 1 in the 9-step process to create the ring signature).

\vskip 0.1 in \noindent This is very useful: it means that the entire set of signatures, both fake and non-fake, are reconstructible by both sides. Note: it doesn't mean you could not bother to publish them to the network (wouldn't that be great!), because then the network has no proof that your transaction values are in range, and not overflowing. But we can still use this `deterministic' nature: we can send data ``over'' these pre-determined signature values.

\vskip 0.1 in \noindent Suppose Alice wants to send the message ``hello" to the receiver of the transaction. After she has constructed the very first signature value (first ring, first digit), let's call it $s_{00}$, using RFC6979 hmac with the ECDH scanning key as seed, she can use xor ($\oplus$) to ``add'' the data for hello; i.e `h' $\oplus s_{00}[0]$, `e' $\oplus s_{00}[1]$, .. etc. XOR is the right way to do this; it preserves the entropy of the original pseudo-random number $s_{00}$ perfectly (in other words, it remains totally random so no one not possessing the seed sees anything unusual).

\vskip 0.1 in \noindent As is mentioned in \cite{ct_wu}, this space (which is 16 * 4 * 32 bytes, or 2048 bytes) can contain any kind of message. However, there is one very specific "message" which \textbf{must} be transmitted - the transaction output amount! 

\vskip 0.1 in \noindent (Note that the signature values themselves, technically, \textbf{do} contain this information from the point of view of the receiver. Because the receiver could look at all the signature values and check which ones were generated by the RFC6979 hmac procedure described above - the ones that aren't \textit{must} be the genuine, non-forged ones, and so must correspond to the correct values of the digits. However, if the receiver deduced the amount this way, he could only do so if none of the forged signatures had been tampered with using XOR; in other words, it would work, but it would preclude sending any other message).

\vskip 0.1 in \noindent So the way CT implements it is this: first, encode the amount of the output as an 8 byte value (as discussed earlier, this is the standard format for a Bitcoin amount - an `int64\_t' which is an 8 byte value). Then, xor this amount in \textbf{3 separate places} into the last of the 64 signatures - so 3 separate fixed places in that 32 byte value. Also, add an xor of the number 128 into the first byte of that signature (to avoid a problem which would occur with zero amounts). However, if the last signature is a non-forged one, it must not be tampered with like this, so in this case Alice the sender will do the same operation with the last-but-one signature in the list.

\vskip 0.1 in \noindent The receiver is then faced with the task of ``undoing'' this embedding operation, to read out the transaction output amount. He starts by doing the same rfc6979 generate operation as the sender, getting the same random values (as before, this generates \textit{both} the blinding factor for each digit, what we called $x_{i}$ for $i=0 \ldots 15$, \textit{and} the forged signature values). Then, he looks at the last signature and does 

\[\textrm{rfc6979 generated random number}\ \oplus\ \textrm{received signature from Alice} \]
\vskip 0.1 in \noindent The result of this xor should look like (here we use our existing example, sending the amount 100000000):

\[ [128,0,0,0,0,0,0,0,0,0,0,0,5,245,225,0,0,0,0,0,5,245,225,0,0,0,0,0,5,245,225,0] \]

The receiver doesn't know in advance that the amount is 100000000, but sees the following pattern: it starts with 128, and then has the same 8 byte string repeated in positions 9..16, 17..24, 25..32: 0000000005f5e100 (in hex), which is 100000000 in decimal. Now suppose the signature is \textbf{not} forged: the chances of such a repeating pattern occurring after xor-ing with the random rfc6979 generated value are essentially zero (this is the reason for 3 repeats - if the value were not repeated, that chance would be decidedly non-zero, remembering that this process is going to be happening for \textbf{every} transaction on the network). So, in this case, the receiver says ``OK, that is clearly the amount, embedded in the last signature by Alice''. If he had tried this and \textbf{hadn't} found the prescribed pattern, he would have moved to the last-but-one signature and tried again (this is what Alice would have done if the last signature had been a non-forged one). So, assuming he was successful, he now knows the amount, and so he also knows the correct value for each digit in the ring signature (remember, it's base 4) - 

\[ 0011331132010000\]

So now he also knows which of the 64 signatures are forged, and which are (claimed to be) real, he can proceed to do the same xor on all the forged signatures (and - a detail mentioned earlier - the $k$ values or nonces for the non-forged ones) and extract any message Alice might have chosen to send along with the transaction. Note that since there's a total of 64*32 bytes involved, and we reserved the last 32 bytes for the amount, that leaves a space of 2016 bytes for this message to take up. This is plenty for attaching some important metadata to a transaction, for auditing purposes or otherwise.

\vskip 0.1 in \noindent Of course, once he has extracted this message, he goes on to verify the ring signature and the Pedersen commitments, just as Nelly the node will have done, to check that the transaction is valid.

\subsection{More details on format, and space saving optimization}

Now that we have the whole system in place, we can understand the structure of a transaction output. A complete set of data for a valid Confidential Transactions output needs three elements: the Pedersen commitment to the output amount (which takes up 33 bytes, a compressed Bitcoin public key), the Borromean ring signature which acts as what we call ``rangeproof'', and the ECDH pubkey which we are calling the 'scanning key' (which takes 32 bytes).

\vskip 0.1 in \noindent The rangeproof is a large and complex structure, so we take an example transaction from the network - id:

\vskip 0.1 in \noindent 6594961e99c7bd271e0aea311bd32a759199c463ce3ac845c6bc7da3c5db2a1d 

\vskip 0.1 in \noindent, and break down the elements in the following tabulation:


\vskip 0.1 in \noindent serValue decoding for transaction 6594961e99c7bd271e0aea311bd32a759199c463ce3ac845c6bc7da3c5db2a1d
, vout index 0.

\vskip 0.1 in \noindent Pedersen commitment for this vout (no leading length byte, optimisation):
\vskip 0.1 in \noindent \textbf{039459f96748d10350009372882d43fc15ab15768931ea2357b3921814f8ec4f18}

\vskip 0.1 in \noindent \textbf{fd040a} - length of the following data, 2564 bytes

\vskip 0.1 in \noindent Range proof begins:

\vskip 0.1 in \noindent \textbf{40} - exponent encoding (exponent=0), minimum value encoding (minimum value = 0)

\vskip 0.1 in \noindent \textbf{1f} - mantissa-1, 31 (so mantissa = 32 bits)

\vskip 0.1 in \noindent \textbf{370e} - 1 sign byte for each 8 rings, here 2 bytes and 16 individual sign bits: 0011011100001110

\vskip 0.1 in \noindent 15 commitment pubkeys, 1 for each ring -1 ; these are the `base'
values each digit/ring which are then expanded using $C' = C -aH$ for
each amount $H$ (the $aH$ are pre-computed to speed up processing).
\vskip 0.1 in \noindent To complete the pubkey parsing, the 32 byte value here is combined
with the sign bit (see above) to give the full pubkey.
The reason that only 15 and not 16 are passed OTW is that one digit
can be worked out by homomorphic subtraction.

\vskip 0.1 in \noindent 
\begin{verbatim}
52e7932b1d5467a595737ad84d19d9d94a40f8d0ed1588e78e6687d5ee2bb05e
7b6710efa9252df2f33026db2ba8d6f61c2b3ce6086de94fffb8f7ca7e9859c4
b3716fbd96c128483e97cbda5b2c4782ae0584cea0d98b06bd8146b510493603
5d90fef716a9f09bcf713f5e3d7191c6bc1dba7144b1faaaeda2f747e62e690c
d50ba524af075208ce10f6c0781688d6aedd1bd3ead5d416d2d21669dd0340b2
22dd1871c6251a53fe74a76de484a3b287be91eae9a8eddb300d1971ea1b4a57
4301a76c00bf54327548b5f8a5ee5c0c2cbfd0df87e0fe566a0f2536daee9cfb
6b381ea6c6e597a718b549262b193d578e6b69bec61998d39a48636e0a2258d4
e31e3919d7ebd91fa302639888c670637e32d2734918a9db4410ed3cd7fc028a
3d5d7422579e55c3235bd497953c82adc752cc72449d72d31482c5d8eab7feb0
c281c56505aefc1451987eb04d725221ae95295a38b5b7809efa814be77f65b7
fb469fb2ee289cffe484b0c9a3caa6a56643941fbd9163b48541850039973704
85cb895c1fef3f9e1f837c6c2e3b819cec5267fb4faedbf5ea7f3b50092a39e6
803236bf686cfb4b22015bb06583ce945b7bf3ad7a37706c90214349ac73a640
129790ee61f645e6b893eb0663a1b1f53ddb2731f80f7c31fa72deeb872d2877
\end{verbatim}

\vskip 0.1 in \noindent The $e$ value ($e_0$) for the borromean ring signature over all the 
digits. 32 byte hash:

\vskip 0.1 in \noindent
\begin{verbatim}
1ba3941b54df283dc1a0b0358cea4fd126df12e8dd03a543de685102329581da
\end{verbatim}


\vskip 0.1 in \noindent Signature values - 32 byte hashes - 16 groups of 4. In each of 
the 16 rings, 1 of the 4 is not forged, and corresponds to the
correct value of the digit in base 4.
\begin{verbatim}
504172e90f65ab7010ab3dd972aaeed28ac98cf25071b63067ae0eb89b337390
db440aed43aea78bcca1aba516aab1cfa9b1ce224d3ebc7492bb1d02596034d8
3df3b337370162ce11b8a077adb533fb8d82d48ec74adf6e4f6f6c6c2af65e49
03cfd7e894631ed0e634a62ce22f342de70be8f59dbe048bbff94329763ad24e

f594a6eba4a403f5ac3d49ffd962eb09a606cee83aecd21e48203518ccee08e4
2f6d1670d1de660ba93638073f8d4910464c28655074895d5fb21af23d25cb22
3d3f4da030f43803be0d7f59a91a16fca841a509c66e9fbe413334b7fad2089f
7bfe61bec8bee18ced3b1bbf0338b7f7e0201246c7cc0a1ddb5fd5013f11e732

f041610547addb7712190195020835b31cfdb498374f10d292faa5af764a1a5d
f570561a7d869719649a2f0a7432eb8ba4a3002816773c054f5f74dc9d10d8ae
db8364a5c326fa8d01fe021a3f078ef2991c175723c888df1eccc1413c2e3f45
50c2e42ef4b6bb5589d9e08e9a0f2f2d31d0c1004e47fee86385b8ab6dc328d4

65ebab5a0ff95fda1aba4fc2012ecaf820338ee46adaf5f468f60e864d591098
f1c01dc9586ca14eb990eabdde889e29cbfc17a6dca7f7a5b2f599c070501fde
614431ef3ebb1865691ed0829c5b82680b3f1a1f7d652752bc3fd07c54595f07
6f802b5eaf3ada0497cafb74574b1e228ae1e3e5a57ffadf4185d9f41dbe63e5

55b19cf6e9c84d6952295b7425ed74ffa1faa1c8e45dec99fce68f1c20092f33
11009e5f5e5d16ab5fdb5713927a46d9db097faa66056927a3434560141d40e2
d5e70f4958ca6579e560563fd44f5099bf4be91feeefe8a11b51a81660814965
6b4058f991a9b001d5f1de584581e31e2c18152ada0ff5b523af95df5566e10d

360116b14d638299c017b74721097c94a8cfdfe6be1f8187a7c899be1ee55d91
3241b5d03bd02f387d6505ff21264a99bbd99dff860528753317556f16af897d
682e8e6760db4c3d68664083f9701910fc448d85750bcae49ae292417d8910e0
1390b310cd95841aafe8f26146ef65bec8b3e86c332a6cfeabbb162fa403aac3

386d09a273f7f523ea17fadc5cee64a926966cb8b9c79426178b5105a88e4247
9a976fca592351797ef742873abf57b67c75a8bdf468f18712e8d1a207960c5e
61de23007a0de3c46f24eb06229b6357b4e7c6351bea4ae80d0e82add09773fc
7aeef8b9d7ae8d166a34cb2a9ae806e98e94bd045062752fc96ca9e13aa0d2c3

f78f858136d22a19584d5c8da46cd782675e3f7ff11b0e682f9b4ccdbececbf6
85a9757d94f6eb41e1c69b0404575da38344adcf88576885ef877bc029f764a8
e2eb097e80e29de0bd005d03db08f8c763ac941ec65fda165d91e64d4c6ca04b
44cceaf5b884f0aa94ca12559859a5701b4ff49a7377c686305ab331b0b0c9d6

7acbd6c7a7518977265f7932c6ab359b66ec6fb88637d2ba8b2cc22d3040febc
951fdd9d7d3ab1a103483f2dd30375027b0257c2b16614ce1ded3aee8e32949e
904f1592c0e15233eb2a5832933675769ccb411022bfc7053e7a9a8ccb976b60
aabdc32ebe367021b55fbc6208311a65cb747864cde551558280de214e9349b6

c05914a328af16e6038985fa5d14519659625e75a50aeb182bc684bc3681b073
6481f0fe0eaffa7095f9fc54a028449697d4788c29720b1b6466cdf65ab888f7
c71bdf5b1934b51b2c85c7e7d451e55291c5cd3c52a1ed00a252b117fd566797
3c8ece6c8379fe1618fd041bd41374f4829cd62c1ff8a0e708677501a23d357a

285dcd441170045a5731a557c3fbff3d85db7762c94f6d0a0c9b5154ec816410
0aed11a1ef39acf75f207af8a066dcabcdc5ea18a58289b36dc8e4dbaf7eb792
632686348ddd67e32feff25a64d527c128804aa199e272947e1beb191e1bd6b8
9fe2db052de3d87de7e0334504942a163055bd2cceb55c37bb91575a46a8e00d

4b14a932d5059bb44747d67a7814b87330457fd33ca71159e514ba8050101be2
77b04c5bdb939f20fc88f18acfb016798e54b66a090f999a7b04766ce48d1d33
b8b1494235da8c7ae4bf3d0cb0565ddaa5b7315dc5b235bcd76df2b74c636c21
719d6f2c6ee04230391a35963b553c866cd6f91daddaedb7303a5f4030cd8725

4a933c3e4ae0fbfb018498309fdbeac66ff09151748c3b7b32336b69d0f4a69e
9d49b6b7ec1ef82945e75749f3cf106bf42fff2a1ffb767ad1dc474eeaf058dc
17b959534d2bf928bbfd0a04cf589d87c8bdd494940bf288b428ba8b840e2d93
bd0dfb936df722fd05f15734cd199e60b179b4e61b02f2ceff1c0c8ef426d610

8ae69a60d78f9cef569de9a3e3587a7b25a896bca2f0514f838c944619334806
0eaa951d1de577ec41f8d5c3ba3ad29063768cebc8c64a7d0308f67db5dc7f67
4cc878e67f6e5d3e899b010bf7e89ac40da55a3b0f1dd656c37d47281683a08d
d823c7bad04c5a81726586dab1a2a95e18580c67a3e9459d34d057b938050370

058454a47f8a7fab7bfdb961646275c87dcea5bbc6cdf96a3c2c2420988461e7
903d715653649fd4bc717967248adedfea96c10abbeffbdf21b560a4e3b3caf4
555769c722dfb29e4c52ccec2c11894d77bff08299f21c46aa8b92399a7b9584
fb1fe37acb3313351550bd23c963af6e0a62dfdbdc09140020766a5329929163

41929b2051798ab1d8b3af57546c76d246befc5e6ba1e948c39013fa87fb6a80
df639046dc03a6ce163df208620d819e210fb752ba1e6403609c4ca123645be5
36982bf728e2f4b47d9b6890f52f097ecf1d1bd4584e018b34a421c1bb9b55c7
89fb063a2097f44356c4a24f76788b28b3ead98f8f3851826ce637a6b31c13ec
\end{verbatim}


\vskip 0.1 in \noindent 21 - length byte, 33

\vskip 0.1 in \noindent The ECDH pubkey for generating the nonce allowing the receiver
to re-generate the random/forged signatures.

\begin{verbatim}
039437c731308c3256b980327e1b806e816e9c01267fe9a17e938864344148a8cf
\end{verbatim}


\vskip 0.1 in \noindent Total bytes for the whole `serValue': 33 + 3 + 2564 + 1 + 33 = 2634 bytes

\vskip 0.1 in \noindent As seen in the comments above, there are 2 optimizations to reduce space usage. First, we do not use 33 bytes for each of the $C_{i}$ Pedersen commitments for each digit. We use only 32, and encode the sign (which would usually be encoded using the byte '02' or '03') inside 1 or more 'sign bytes' at the start. Each sign byte can encode 8 sign bits, so in the default case of 16 rings for a 32 bit value, we use 2 sign bytes. The second optimisation is a feature of the Pedersen commitments. Remembering that this single transaction output has its overall Pedersen commitment, $C$, we always must have:
\[C = C_{0} + C_{1} + ... + C_{15}\]
\vskip 0.1 in \noindent Because of this, the last value $C_{15}$ can be considered implicit, and deduced both by the network and the receiver, by subtracting the previous 15 values from the starting value $C$. That's why the rangeproof contains only 15, rather than 16, commitments for the digits.

\vskip 0.1 in \noindent The upshot of this is that a typical transaction output needs ~ 2.5KB. So a typical CT transaction will take somewhere around 10KB of space.

\subsection{The scanning key}

Now that you can see the whole picture, it's easy to understand that if Alice or Bob decide to, they can share the nonce generated by the ECDH key exchange process, thus exposing the values and messages embedded in the transaction with other parties such as auditors. And they can do this without, of course, sharing the private keys for spending. This is clearly a very useful feature - selective privacy.


Code references:
Most of the code for Confidential Transactions can be found in 

\begin{verbatim}
src/secp256k1/src/{rangeproof_impl.h,borromean_impl.h}
and 
src/{transaction.cpp, blind.cpp}
\end{verbatim} 


\pagebreak

 \begin{thebibliography}{1}

  \bibitem{ct_wu} Confidential Transasctions write up by G. Maxwell: https://people.xiph.org/\~greg/confidential\_values.txt
  \bibitem{chaum_dc} http://www.cs.elte.hu/\~rfid/dcrypt\_chaum.html
  \bibitem{ms_ringsig} https://research.microsoft.com/en-us/um/people/yael/publications/2006-leak\_secret.pdf
  \bibitem{borromean} https://github.com/Blockstream/borromean\_paper/
  \bibitem{pbtc} https://github.com/vbuterin/pybitcointools
  \bibitem{rfc6979} Deterministic usage of the Digital Signature Algorithm: https://tools.ietf.org/html/rfc6979
  \bibitem{hmac} Wikipedia's description of HMAC: https://en.wikipedia.org/wiki/Hash-based\_message\_authentication\_code
  \end{thebibliography}
  

\end{document}





	










































